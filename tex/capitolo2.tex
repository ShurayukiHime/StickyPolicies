% !TeX spellcheck = en_GB
\chapter{Sticky Policies in literature}
\label{chapter2}
\thispagestyle{empty}

To implement \textit{Sticky Policies} we first focus on the context for application. In a simple use-case, user Alice wants to share some personal data with user Bob through an application, which we will call \textit{Service Provider}. We can generalize this situation for all those cases in which a service provider is fed with some personal data in order to supply the data owner with a service. To protect her privacy, Alice will specify which entities are allowed to use her data, and for which purposes through a \textit{Sticky Policy}. The policy will be attached to the data so as to stick with them persistently, and the data will be obfuscated to prevent unauthorized access.

Most of the suggested solutions regarding this issue share a similar low-level architecture. Thus, we can recognize three basic common entities, which are essential to the development of \textit{Sticky Policies}. They are the following: 
 \begin{itemize}
 	\item The \textit{Data Owner} has a finite collection of data, which she wants to protect through fine-grained policies.
 	\item A trusted entity which is in charge of securely storing data and generating encryption keys.
 	\item A \textit{Service Provider} should be considered as third party which requests data usage.
 \end{itemize}
The role and implementation of the trusted entity changes according to the chosen encryption scheme and protocol.

We will now examine the main solutions available in literature, and consider their main advantages and disadvantages. It is worth saying beforehand that every solution cannot overlook the trust in the chosen third party: mainly, because once the data is decrypted it can be shared by the third party without any concern; secondly, due to the need of proving the remote hardware machines to be reliable (i.e. it always behaves the way it should, for the intended purpose). This latter issue has been dealt with through the use of Trusted Platform Modules \cite{standard2009trusted}.

\section{Hybrid Cryptosystem}
In this scenario, we use both asymmetric and symmetric cryptography to achieve the required \textit{stickyness} of the policies. In a possible use-case, the service provider receives data, encrypted with a symmetric one-time-use key \textit{K}, together with the policy and \textit{K}, both encrypted with the public key of the \textit{Trust Authority} and signed by the user. The service provider will interact with the Trust Authority to prove its reliability, eventually receiving the symmetric key \textit{K}.

The \textit{Trusted Authority} or \textit{Policy Enforcement Point} always mediates the data exchange. It is a semi-trusted third party which checks the compliance of the service provider with the specified policies before releasing the symmetric keu \textit{K}. A formal protocol for message exchange is suggested in \cite{pearson2011sticky}: \texttt{Policy, Enc(PubTA, K||h(Policy)), Sig(PrivUser, Enc(PubTA, K||h(Policy))), Enc(K, PII)}. This ensures that the policy always sticks to its data, and its integrity can be verified through a secure hashing function. Moreover, the combined usage of TA's public key and the data owner's private key ensure both confidentiality and authenticity. 

This implementation relies heavily on the Public Key Infrastructure, and requires procedures for management and verification of X.509 certificates. For what concerns the PEP, it must be always reachable from the internet and, together with the Certification Authority, may be a target for attacks: first, it constitutes a \textit{single point of failure}; additionally, if compromised, it could infect the data owner (for example through phishing attacks) or act as a man in the middle, decrypting \texttt{PII}.

In both cases, this solution proves to be computationally heavy and it does not solve the main issue of trusting the service provider not to illegitimately share data.

\section{Attribute-Based Access Control}
The Attribute-Based Access Control (ABAC) paradigm well fits our environment: by describing users and objects through a set of attributes, it allows fine-grained policy specifications for data protection. However, it requires a precise definition of the descriptive attributes and the implementation of the architecture required to process and enforce policies.

The XACML standard \cite{standard2005extensible} provides a valid reference by defining not only the attributes and structure for rules, but also the components in the architecture. For the sake of simplicity, suppose we need only the \textit{Policy Enforcement Point} from the whole XACML standard implementation, since we assume that the data owner has a client-side module to generate the required XML policies. The data owner trusts the \textit{Policy Enforcement Point} to be both a secure storage for its personal data and to evaluate correctly the reliability and compliance of the service provider. %data can be stored encrypted, but having the data stored in different physical systems (in a distributed system) lowers the security level

This solution allows a better data description in terms of granularity, and it is also more efficient once the architecture has been implemented; nonetheless, it shows the same weaknesses as the Hybrid Cryptosystem. Additionally, the effort required for architecture implementation and rule generation should not be underestimated. It is also remarkable that the mobile environment evolves quickly and is exceptionally fragmented and varied, conditions which make it more difficult to establish fixed descriptive attributes.

In this context, we can also consider Ciphertext-Policy Attribute-Based Encryption \cite{bethencourt2007ciphertext}. This solution does not require an intermediate entity to evaluate the policies before forwarding sensitive data to the recipient, because this assessment is embedded in the cryptographic scheme so that no unauthorized individual can decrypt it. In particular, the data owner chooses a set of attributes defining an \textit{access tree}, the structure used at a lower level to bind attributes to the ciphertext and ensure that only individuals that satisfy the access tree can decrypt the obfuscated text.

This solution requires an available trusted entity for key generation, but does not strictly need a \textit{Policy Enforcement Point}. It is possible to leverage on a secure storage to ease access from service providers, even though the level of trust required from this third entity is low due to data being encrypted by the data owner. As highlighted by Tang  \cite{tang2008using}, a drawback of CP-ABE occurs when a private key is compromised and it is necessary to issue a new one: there is a non-negligible amount of risk related to the possibility for a potential attacker to decrypt all the ciphertexts associated with the attribute set of the compromised key.

\section{Identity-Based Encryption}
The Identity-Based Encryption (IBE) paradigm proposed by Shamir \cite{shamir1984identity} can be purposely used to realize \textit{Sticky Policies} \cite{mont2003towards}.

IBE is an asymmetric cryptographic scheme which eliminates the need of the Public Key Infrastructure together with X.509 certificates, requiring only a trusted entity to generate private keys when needed. The public key should be an identifying, non-repudiable attribute, e.g. the email address: when an encrypted text is received, the recipient asks the key-generation centre to issue the corresponding private key, thus being able to decrypt the text.

In \cite{mont2003towards}, Mont describes a cryptographic scheme for \textit{Sticky Policies} based on IBE and coupled with a Trusted Platform Module. The encryption key is a XML document containing the formalization of the policy, thus reaching the desired \textit{stickyness} for \textit{Sticky Policies}. Any tampering or policy non-compliance will prevent the Trust Authority from generating the correct decryption key. Additionally, the Trust Authority will check the reliability of the requester before issuing the decryption key. 

Finally, as shown by Shamir \cite{shamir1979share}, Mont suggests to increase the security level through a \textit{threshold scheme}, involving several Trust Authorities for key issue. In a \textit{(k,n)} threshold scheme, the key is divided into \(2k-1\) parts, and at least \textit{k} pieces are necessary to encrypt or decrypt: as a disadvantage, it proves to be less efficient than a simple IBE scheme.

Tang \cite{tang2008using} suggests a similar solution, in which the key is a string containing the identity of the recipient concatenated with some attributes or constraints, e.g. time stamps, so to make IBE finer-grained.

Both of the approaches require a new key generation every time the policy is changed, which could be inefficient in contexts as mobile devices communication; moreover, even if they do not directly use attributes in the encryption, there is still the need of a standard definition. When using Mont's solution, an XML grammar specification is necessary to avoid generating different keys for equal policies improperly written, while in Tang's case a specification should be issued to determine which attributes to use and their order.

\section{Proxy Re-Encryption}
The Proxy Re-Encryption scheme \cite{green2007identity} can also be taken into consideration as an implementation for \textit{Sticky Policies}, and is particularly suited to the mobile environment.

Communication and data sharing through mobile devices is supported by remote servers instead of happening point-to-point. In this context, the remote server is the Proxy, which re-encrypts data from the sender Alice's signature to the receiver Bob's. The server can be untrusted, as the scheme never produces plain text and is unidirectional and resistant to collusion attacks. To implement \textit{Sticky Policies}, Alice sends a policy with the encrypted data: a \textit{Policy Enforcement Point} evaluates Bob's policy compliance, and data is re-encrypted and forwarded with the re-encryption key generated by the server.

The Proxy Re-Encryption scheme can be implemented either through Identity Based Encryption or Public Key Encryption, and in this latter case provides the benefit of Alice encrypting only for the Proxy, which will then be in charge of re-encrypting with a different key for every recipient.

The Proxy Re-Encryption scheme shows some advantages with respect to the other schemes mentioned: key and policy updating are performed by informing the proxy, and no vulnerability affects previous ciphertexts encrypted with those keys or conditions thanks to the re-encryption. Disadvantages of PRE are highlighted in \cite{tang2008using}: even though the scheme requires a lower level of security, a compromised Proxy could generate re-encryption keys for any receiver, potentially exposing personal data to any of its recipients. To address this issue, Tang presents the Type-based PRE, which introduces the notion of \textit{data categories}, as an additional input parameter to the re-encryption key generation. With TPRE, compromise of a re-encryption key does not affect keys with a different type, and key revocation is dealt with just creating a new re-encryption key relating to a new data type (which includes the previous one).

\section{Existing implementations and libraries}
Nearly each of the possibilities considered in chapter \ref{chapter2} has a dedicated implementation. For Cyphertext-Policy Attribute-Based Encryption there is the \texttt{cpabe} toolkit \cite{bethencourt2011library}, available in the C language. This library provides an encryption scheme such that each private key is associated with a set of attributes rather than with the identity of the data owner. Attributes can be provided as strings from standard input or from a file; moreover, it is possible to combine more than one attribute or rule through predefined operators as 'and', 'or', '>', '<'.

\subsection{Java Pairing Based Cryptography Library for IBE}
For Identity-Based Encryption we can rely on the Java Pairing Based Cryptography Library \cite{ISCC:DecIov11}. Private keys are generated from identities alone as well as combined with attributes describing the authorized audience, and it is possible both do encrypt and sign data. Both \cite{ISCC:DecIov11} and \cite{bethencourt2011library} depend on the Pairing-Based Cryptography Library \cite{PBC2007Lynn}, developed in the C language.

\lstinputlisting[caption={Java mock class for IBE implementation},label={AHIBEDIP10},language=java]{AHIBEDIP10.java}

As shown in Listing \ref{AHIBEDIP10}, the private key is generated providing several strings in place of the data owner's identity. Different cyphertexts are produced using different attributes as a key, and the same \texttt{CipherParameters} are needed to decrypt correctly.

The functions \texttt{encaps} and \texttt{decaps} provide encryption and decryption mechanisms. After practical experiments, it results that \cite{ISCC:DecIov11} is a proof-of-concept implementation and it is thus not suited for actual use. The main reasons behind this lay in the implementation of the aforementioned functions: in fact, the \texttt{encaps} function does not take any plaintext in input, but it is generated inside its body by the function \texttt{process()}. As we can see from Listing \ref{PairingKeyEncapsulationMechanism}, this function calls \texttt{processBlock} supplying as input an empty byte array instead of an actual input.

\lstinputlisting[caption={Excerpt from PairingKeyEncapsulationMechanism class},label={PairingKeyEncapsulationMechanism},language=java]{PairingKeyEncapsulationMechanism.java}

It is possible to modify the source code by opening a file, or supplying a run-time byte array containing the information to encrypt, and calling \texttt{processBlock} purposely:

\lstinline[language=java]!return processBlock(dataArray, 0, dataArray.length);!

To obtain the encrypted text it is also necessary to modify the last statement in the \texttt{processBlock} function called by \texttt{process}. In fact, as shown in the documentation for class \texttt{PairingAsymmetricBlockCipher}, the function \texttt{byte[] processBlock(byte[] in, int inOff, int inLen)} takes as second and third arguments the offset and the length of data, thus requiring an invocation like the following:

\lstinline[language=java]!return processBlock(in, 0, in.length);!

The complete content of the mentioned classes is available in Appendix \ref{appendixB}.

After performing data encryption, though, the \texttt{Assert} statements to verify correct decryption fail, which leads us to think that this is only a proof-of-concept implementation. For this reason, and due to the unsuitableness of the available \cite{bethencourt2011library} and \cite{PBC2007Lynn} in the chosen context for this project, we have thus decided to proceed to the implementation of Sticky Policies with a hybrid cryptosystem.