\chapter{Getting one's hands dirty}
\label{chapter3}
\thispagestyle{empty}

\noindent First, we set up the necessary entities to implement \textit{Sticky Policies}: an Android client and a service provider. The client was developed using Android Studio, while for the service provider, we used Eclipse together with Apache Tomcat; the communication was implemented through HTTP protocol.

// ADD SOME TECHNICAL DETAILS

\section{Implementing Sticky Policies through XML}
The first experiment involved realizing \textit{Sticky Policies} via XML files. This approach was motivated by \cite{mont2003towards}, and an example XML policy was created taking as a reference the one presented in the same paper.

In this scenario, two entities, an Android client and a server, communicate between each others and two different kinds of protocols are possible. 

To ensure compatibility, we rely on an XML grammar defined through a XSD file.

Server side, a the received policies are parsed and checked for compliance; ADMITTED policies

 an XML grammar was specified through a XSD file, and a parser was implemented server-side to check the received policy files for compliance; default behaviour for non-compliant policy was disclosure refusal.

More specifically,