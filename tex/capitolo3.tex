\chapter{Getting one's hands dirty}
\label{chapter3}
\thispagestyle{empty}

\noindent First, we set up the necessary entities to implement \textit{Sticky Policies}: an Android client and a service provider. The client was developed using Android Studio and emulated through the Android Emulator with a Nexus 5X device running Android 7.0 and API level 24. The Trusted Authority was developed using Eclipse and run on Apache Tomcat 8; the communication was implemented through HTTP protocol.

\section{Existing implementations and libraries}
Nearly each of the possibilities considered in chapter \ref{chapter2} has a dedicated implementation. For Cyphertext-Policy Attribute-Based Encryption there is the \texttt{cpabe} toolkit \cite{bethencourt2011library}, available in the C language. This library provides an encryption scheme such that each private key is associated with a set of attributes rather than with the identity of the data owner. Attributes can be provided as strings from standard input or from a file; moreover, it is possible to combine more than one attribute or rule through predefined operators as 'and', 'or', '>', '<'.

\subsection{Java Pairing Based Cryptography Library for IBE}
For Identity-Based Encryption we can rely on the Java Pairing Based Cryptography Library \cite{ISCC:DecIov11}. Private keys are generated from identities alone as well as combined with attributes describing the authorized audience, and it is possible both do encrypt and sign data. Both \cite{ISCC:DecIov11} and \cite{bethencourt2011library} depend on the Pairing-Based Cryptography Library \cite{PBC2007Lynn}, developed in the C language.

\lstinputlisting[caption={Java mock class for IBE implementation},label={AHIBEDIP10},language=java]{AHIBEDIP10.java}

As shown in Listing \ref{AHIBEDIP10}, the private key is generated providing several strings in place of the data owner's identity.

The functions \texttt{byte[][] encaps(CipherParameters publicKey, Element... ids)} and \texttt{byte[] decaps(CipherParameters secretKey, byte[] cipherText)} provide encryption and decryption mechanisms. After practical experiments, it results that \cite{ISCC:DecIov11} is a proof-of-concept implementation and it is thus not suited for actual use. The main reasons behind this lay in the implementation of the aforementioned functions: in fact, the \texttt{encrypt} function calls \texttt{byte[] processBlock(byte[] in)} shown in Listing \ref{PairingKeyEncapsulationMechanism}, which calls, in turn, \texttt{byte[] processBlock(byte[] in, int inOff, int inLen)}, shown in Listing \ref{PairingAsymmetricBlockCipher}.

\lstinputlisting[caption={Excerpt from PairingKeyEncapsulationMechanism class},label={PairingKeyEncapsulationMechanism},language=java]{PairingKeyEncapsulationMechanism.java}

text text text

\lstinputlisting[caption={Excerpt from PairingAsymmetricBlockCipher class},label={PairingAsymmetricBlockCipher},language=java]{PairingAsymmetricBlockCipher.java}

\section{Realizing a Hybrid Cryptosystem}
In this scenario, \textit{Sticky Policies} are realized via XML files and paired through the combination of symmetric and asymmetric cryptography with some personal data. This approach is presented in \cite{pearson2011sticky}, and an example XML policy was created taking as a reference the one presented in the same paper. Following this specification, we present a protocol for the communication of two Android clients, called for simplicity Alice and Bob. 

Alice generates an XML policy to regulate data access, encrypting it with a symmetric key generated locally. The policy and the keys are then encrypted with the Trusted Authority's public key and signed by Alice. For this purpose, Alice should obtain the public key of the Trusted Authority and also a key pair for herself: in our solution, we use self-signed X509 Certificates generated by combining the Java Cryptography Architecture and the Bouncy Castle cryptography APIs for Java. Purposely, Alice contacts the Trusted Authority to obtain its public key, and shares her own if the Trusted Authority is not the issuer of the certificate itself.

To decrypt Alice's data, Bob has to follow this steps:
\begin{itemize}
	\item Bob obtained Alice's encrypted data and an attached policy in clear text, together with other encrypted data to guarantee integrity, confidentiality and non-refusal of the policy and data.
	\item Bob asks the Trust Authority to release the symmetric key, presenting the encrypted data signed by Alice.
	\item The Trust Authority evaluates the policy and Bob's reliability, submitting some challenges for him to complete.
	\item If trusted, Bob receives the symmetric key to decrypt Alice's personal data.
\end{itemize}

Data is sent through POST requests over a channel which is assumed to be secure.

Once the Trust Authority receives a request from the client, it checks the correct specification of the policy before proceeding to decrypt and verify the payload received. This operation is performed by a server-side parser which matches the XML file with a standard XSD grammar. In case of errors, no symmetric key is released.

The specification of the XSD grammar can be found in Appendix \ref{appendixA}. 

\subsection{Structure of an XML policy}
The policy file is constructed by the data owner in order to specify which set of users can access her data. 