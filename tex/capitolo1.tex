\chapter{Introduction}
\label{Intro}
\thispagestyle{empty}

\noindent 
Personal data increasingly fuel Internet applications development and spread, especially since the birth and diffusion of Big Data. Users are often unaware of the gathering, processing and storage of their data, mostly because of the obfuscation behind license agreements and the technical notions necessary to understand these processes. Data may also be moved across country borders, to be processed under different laws and regulations, with processing systems so complex to ultimately make it impossible to understand which service provider had the right to process what.

Leaving out the risks originating for accidental data disclosure by such applications, for example due to security breaches, users may wish to increase their data protection, especially in fields as medical care and financial services. More specifically, we would like a tool to selectively reduce data disclosure, preventing unwanted usage from third parties.

Current solutions include state laws about data usage and protection, business frameworks and service-level agreements, which prove to be inefficient and ineffective. Many research studies have thus advanced the suggestion of a different tool, called \textit{Sticky Policies}, to ensure data protection and control their disclosure. \textit{Sticky Policies} are essentially machine-readable metadata which specify the correct usage of the data they travel with \cite{pearson2011sticky}: through encryption, they prevent policy non-compliant use.

In this study, we focus on the mobile environment, and in particular on smartphones. A few solutions already exist, but often they fail to protect personal data with the desired granularity. Moreover, services and application run in remote cloud systems, and data are stored in distributed systems managed through complex automated procedures. Due also to the difficulty in complying with strict Data Regulation laws (e.g. GDPR), many companies are \textit{outsourcing} this task through Single Sign-On procedures: personal data are gathered and processed by large and structured organizations, and external services rely on simple APIs for user authentication.

In general, any communication via smartphone flows through a server or a cloud and thus it would be more realistic to implement a solution which integrated with an open-source app or service adding a layer of data protection. At a high level, we could say that before sharing data from an app, the sender could select any condition to be verified before the recipient could access that data on the same app. In case of non-compliance, simply the data wouldn't be disclosed, while in case the access was granted, it would be possible to prevent illegal data sharing outside the app.

We would like to provide a proof-of-concept solution aimed, in particular, at Android devices. This context is exceptionally varied and quickly evolving, features which pose a limit to the extensiveness of the proposed solution. Other limitations derive from the availability of cryptographic implementations of the studied cryptography schemes, and of open-source mobile applications.

In this project work, we start by analysing the most recent and relevant proposals about this matter, comparing different technical approaches and existing solutions. Then, we introduce a solution of our own, which tackles this issue within a limited scope of action. Finally, we present our conclusions about this work.

\iffalse
- internet funziona sempre più producendo risultati a partire da dati personali, e questa tendenza sta aumentando grazie al diffondersi di big data
- gli utenti sono, nella maggior parte dei casi, ignari dell'uso che viene fatto dei loro dati personali, e in ogni caso, spesso, tali servizi avvengono sotto legislazioni diverse, o avvengono per conto di ulteriori terze parti che potrebbero non rispettare i permessi inizialmente accordati.
- questi costituiscono dei rischi per la violazione della privacy e in generale potrebbero costituire una vulnerabilità, esposizione, degli utenti agli attacchi su internet.
- vorremmo che fosse possibile per gli utenti scegliere selettivamente cosa condividere, dando loro la possibilità di limitare la diffusione dei loro dati personali dovuta a servizi esterni.
- per fare ciò si propone l'uso di sticky policies, ossia metadati che restano agganciati ai dati a cui si riferiscono e danno indicazioni sul loro corretto utilizzo.
\fi